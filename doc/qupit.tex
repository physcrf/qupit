% Created 2019-12-19 四 17:43
% Intended LaTeX compiler: pdflatex
\documentclass[11pt]{book}
\usepackage[top=1in,bottom=1in,left=1in,right=1in]{geometry}
\usepackage{graphicx}
\usepackage{amsmath}
\usepackage{amssymb}
\usepackage{bm}
\usepackage{times}
\usepackage{physics}
\usepackage{subfigure}
\usepackage[colorlinks=true,linkcolor=blue,urlcolor=blue,citecolor=blue]{hyperref}
\usepackage{minted}
\setminted{linenos}
\date{}
\title{}
\hypersetup{
 pdfauthor={crf},
 pdftitle={},
 pdfkeywords={},
 pdfsubject={},
 pdfcreator={Emacs 26.3 (Org mode 9.1.9)}, 
 pdflang={English}}
\begin{document}

\title{QUPIT: QUasiadiabatic propagator Path Integral Toolkit}
\author{Ruofan Chen}
\maketitle
\tableofcontents

\chapter{General Formulation of QUAPI}
\label{sec:org611f55b}
In this chapter we give a review of the general formalism of the QUAPI
\cite{makarov1994-path,makri1995-numerical,segal2010-numerically}.  Let
us consider a generic many-body system which can be modeled by a
finite system of interest coupled to a bath. Let \(H(t)\) denote the
total Hamiltonian which can split into three parts:
\begin{equation}
H(t)=H_S(t)+H_B+H_{SB},
\end{equation}
where \(H_S(t)\) is the Hamiltonian of the system of interest, \(H_B\) is
the Hamiltonian of the bath and \(H_{SB}\) represents the coupling
between the system and the bath. Here we consider the case where only
the system Hamiltonian \(H_S(t)\) is time-dependent. Let \(\rho(t)\) be
the total density matrix, then the time evolution of \(\rho(t)\) is
given by
\begin{equation}
\rho(t)=U(t)\rho(0)U^{\dag}(t),
\end{equation}
where
\begin{equation}
U(t)=\mathrm{T}\exp[-i\int_0^t H(\tau)\dd{\tau}]
=\lim_{\delta t\to0}\prod_{t_i=0}^t e^{-iH(t_i)\delta_t}.
\end{equation}
Here \(\mathrm{T}\) is the chronological ordering symbol, and the
product is understood in that we take the limit over all the
infinitesimal intervals \(\delta t\). Therefore we can write the density
matrix \(\rho(t)\) as
\begin{equation}
\rho(t)=\lim_{N\to\infty}e^{-iH(t_N)\delta t}\cdots e^{-iH(t_0)\delta t}
\rho(0)e^{iH(t_0)\delta t}\cdots e^{iH(t_N)\delta t}
\end{equation}
for \(t_0=0\) and \(\delta t=t/N\).

Now we introduce the reduced density matrix of the system
\(\rho_B(t)=\Tr_B[\rho(t)]\), which is obtained by tracing the total
density matrix over the bath degrees of freedom, then the time
evolution of \(\rho_B(t)\) is given by 
\begin{equation}
\rho_S(s'',s';t)=\Tr_B\bra{s''}
\lim_{N\to\infty}e^{-iH(t_N)\delta t}\cdots e^{-iH(t_1)\delta t}
\rho(0)e^{iH(t_1)\delta t}\cdots e^{iH(t_N)\delta t}\ket{s'}.
\end{equation}
For numerical evaluation we can employ finite \(\delta t\) in the above
expression which approximates the evolution operator \(U(t)\) into a
product of finite \(N\) exponentials. Inserting the identity operator
\(\int\ket{s}\bra{s}\dd{s}\) between every two exponentials and
relabeling \(s'',s'\) as \(s_N^+,s_N^-\) gives
\begin{equation}
\begin{split}
\rho(s_N^+,s_N^-;t)=&\int\dd{s}_0^+\cdots\dd{s_{N-1}^+}\int\dd{s_0^-}\cdots\dd{s_{N-1}^-}\\
&\quad\Tr_B[\mel{s_N^+}{e^{-iH(t_N)\delta t}}{s_{N-1}^+}\cdots
\mel{s_1^+}{e^{-iH(t_1)\delta t}}{s_0^+}\\
&\qquad\qquad\times\mel{s_0^+}{\rho(0)}{s_0^-}
\mel{s_0^-}{e^{iH(t_1)}\delta t}{s_1^-}\cdots
\mel{s_{N-1}^-}{e^{iH(t_N)\delta t}}{s_N^-}]
\end{split}
\end{equation}

\chapter{Reference}
\label{sec:org93c72ee}
\bibliographystyle{plainurl}
\bibliography{/home/crf/Nutstore/Literature/papers}
\end{document}